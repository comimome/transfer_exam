

%% パッケージ
\usepackage{titlesec} % sectionの見た目を編集
\usepackage{color} % 文字・背景の色を編集
\usepackage{url} % URLを出力
\usepackage{amsmath,amssymb} % 数式全般
\usepackage{mathtools}
\usepackage[italicdiff]{physics}
\usepackage[dvipdfmx]{graphicx} % 図の挿入
\usepackage{pgfplots}
\usepackage{tikz} % 図の描画
\usetikzlibrary{intersections,calc,arrows.meta} % TikZのライブラリを追加


%% sectionの修飾
% 四角+下線
\titleformat{\part}[block]
{}{}{0pt}
{
  \colorbox{black}{\begin{picture}(0,10)\end{picture}}
  \hspace{0pt}
  \normalfont \Large\bfseries
  \hspace{-4pt}
}
[
\begin{picture}(100,0)
  \put(3,18){\color{black}\line(1,0){300}}
\end{picture}
\\
\vspace{-30pt}
]
\renewcommand{\thesection}{\textbf{問題\Roman{section}}}
\renewcommand{\thesubsection}{\textbf{問\arabic{subsection}}}
\renewcommand{\thesubsubsection}{\textbf{(\alph{subsubsection})}}


% 増減表の矢印
\newcommand{\nerarrow}{
\begin{tikzpicture}[scale=0.3,baseline=0.3]
  \draw[->,>=stealth] (0,0) to[bend right=45] (1,1);
  \end{tikzpicture}
  }
\newcommand{\nelarrow}{
\begin{tikzpicture}[scale=0.3,baseline=0.3]
\draw[->,>=stealth] (0,0) to[bend left=45] (1.2,1);
\end{tikzpicture}
}
\newcommand{\selarrow}{
\begin{tikzpicture}[scale=0.3,baseline=0.3]
\draw[->,>=stealth] (0,1) to[bend left=45] (1,0);
\end{tikzpicture}
}
\newcommand{\serarrow}{
\begin{tikzpicture}[scale=0.3,baseline=0.3]
\draw[->,>=stealth] (0,1) to[bend right=45] (1.2,0);
\end{tikzpicture}
}