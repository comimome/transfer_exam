\section{令和5年度 数学}

\subsection{}

\subsubsection{}
ベクトル $\overrightarrow{\mathrm{AB}}$ を求め,その大きさを計算する.$\overrightarrow{\mathrm{AB}}$ は
\begin{align*}
  \overrightarrow{\mathrm{AB}} = \begin{pmatrix}4 \\ 5 \\ -2 \end{pmatrix} - \begin{pmatrix}2 \\ 3 \\ -1 \end{pmatrix} = \begin{pmatrix}2 \\ 2 \\ -1 \end{pmatrix}
\end{align*}
となる.よって
\begin{align*}
  \lvert\overrightarrow{\mathrm{AB}}\lvert = \sqrt{2^2 + 2^2 + (-1)^2} = \sqrt{10}
\end{align*}
である.


\subsubsection{}
\begin{tikzpicture}
  \coordinate (A) at (2,3,-1) node [below left] at (A) {A};
  \coordinate (B) at (4,5,-2) node [below right] at (B) {B};
  \draw [name path=AB][thick](A)--(B);
  \draw [name path=cir_A][thick](A) circle[radius=1];
  \draw [name path=cir_B][thick][dotted](B) circle[radius=3];
  \path [name intersections={of=AB and cir_A, by={C}}];
  \fill (C) node [above] {C};
  \foreach\P in {A,B,C} \fill[black](\P)circle(0.06);
\end{tikzpicture}


\subsubsection{}
\subsubsection{}

\subsection{}

\subsection{}