\section{令和5年度 数学}


\subsection{}


\subsubsection{}

ベクトル $\overrightarrow{\mathrm{AB}}$ を求め,その大きさを計算する.$\overrightarrow{\mathrm{AB}}$ は

\begin{gather*}
  \overrightarrow{\mathrm{AB}} = \begin{pmatrix}4 \\ 5 \\ -2 \end{pmatrix} - \begin{pmatrix}2 \\ 3 \\ -1 \end{pmatrix} = \begin{pmatrix}2 \\ 2 \\ -1 \end{pmatrix}
\end{gather*}

となる.よって

\begin{gather*}
  \lvert\overrightarrow{\mathrm{AB}}\lvert = \sqrt{2^2 + 2^2 + (-1)^2} = \sqrt{9} = 3
\end{gather*}

である.


\subsubsection{}

\begin{tikzpicture}
  \coordinate (A) at (2,3,-1) node [below left] at (A) {A};
  \coordinate (B) at (4,5,-2) node [below right] at (B) {B};
  \draw [name path=AB][thick](A)--(B);
  \draw [name path=cir_A][thick](A) circle[radius=1];
  \draw [name path=cir_B][thick][dotted](B) circle[radius=3];
  \path [name intersections={of=AB and cir_A, by={C}}];
  \fill (C) node [above] {C};
  \foreach\P in {A,B,C} \fill[black](\P)circle(0.06);
\end{tikzpicture}

線分 $\mathrm{AB}$ と球面 $\mathrm{\alpha}$ の交点を $\mathrm{C}$ とおく.
球面 $\mathrm{\beta}$ が球面 $\mathrm{\alpha}$ と共有点を持つ条件は問1より以下のようになる.

\begin{gather*}
  \lvert\overrightarrow{\mathrm{AB}}\lvert + \lvert\overrightarrow{\mathrm{AC}}\lvert  \ge \mathrm{r} \ge \lvert\overrightarrow{\mathrm{AB}}\lvert - \lvert\overrightarrow{\mathrm{AC}}\lvert \\
  3 \ge \mathrm{r} \ge 2
\end{gather*}


\subsubsection{}

\begin{tikzpicture}
  \coordinate (A) at (2,3,-1) node [below left] at (A) {A};
  \coordinate (B) at (4,5,-2) node [below right] at (B) {B};
  \draw [name path=AB][thick](A)--(B);
  \draw [name path=cir_A][thick](A) circle[radius=1];
  \draw [name path=cir_B][thick](B) circle[radius=3];
  \path[name intersections={of=cir_A and cir_B, by={D, E}}];
  \fill [black](D) node [above right]{D};
  \fill [black](E) node [above right]{E};
  \draw [name path=DE][thick][dotted](D)--(E);
  \path [name intersections={of=AB and DE, by={F}}];
  \foreach\P in {A,B,D,E,F}\fill[black](\P)circle(0.06);
  \fill [black](F) node [above]{F};
\end{tikzpicture}

円 $\mathrm{S}$ は $\lvert\mathrm{DF}\lvert$ を半径に持つため,

\begin{gather*}
  {\lvert\mathrm{DF}\lvert}^2\pi = \frac{5\pi}{9} \\
  {\lvert\mathrm{DF}\lvert}^2 = \frac{5}{9}
\end{gather*}

となる.また図から次のような関係が成り立つ.

\begin{gather*}
  {\lvert\mathrm{AD}\lvert}^2 = {\lvert\mathrm{AF}\lvert}^2 + {\lvert\mathrm{DF}\lvert}^2 \\
  {\lvert\mathrm{BD}\lvert}^2 = {\lvert\mathrm{BF}\lvert}^2 + {\lvert\mathrm{DF}\lvert}^2
\end{gather*}

$\lvert\mathrm{AD}\lvert = 1$,$\lvert\mathrm{BD}\lvert = \mathrm{r}$,
$\lvert\mathrm{BF}\lvert = \lvert\mathrm{AB}\lvert - \lvert\mathrm{AF}\lvert = \sqrt{10} - \lvert\mathrm{AF}\lvert$
であるため,上2式は次のようになる.

\begin{gather*}
  1 = {\lvert\mathrm{AF}\lvert}^2 + \frac{5}{9} \\
  {\mathrm{r}}^2 = \left\{3 - \lvert\mathrm{AF}\lvert\right\}^2 + \frac{5}{9}
\end{gather*}

整理すると

\begin{gather*}
  {\mathrm{r}}^2 = \left\{3 - \sqrt{1 - \frac{5}{9}}\right\}^2 + \frac{5}{9} = 6 \\
  \mathrm{r} = \sqrt{6}
\end{gather*}

となる.

\subsubsection{}

円 $\mathrm{S}$ の中心座標は点 $\mathrm{F}$,
円 $\mathrm{S}$ を含む平面の方程式の法線ベクトルはベクトル 
$\overrightarrow{\mathrm{AB}}$ に等しい.
点 $\mathrm{F}$ は線分 $\lvert\mathrm{AB}\lvert$ を $1\colon\mathrm{r}$ に内分する点であるため,点 $\mathrm{F}$ の座標は,

\begin{gather*}
  \left\lparen {\frac{\mathrm{r} \cdot 2 + 1 \cdot 4}{1 + \mathrm{r}}} {,} {\frac{\mathrm{r} \cdot 3 + 1 \cdot 5}{1 + \mathrm{r}}} {,} {\frac{\mathrm{r} \cdot (-1) + 1 \cdot (-2)}{1 + \mathrm{r}}} \right\rparen
  =
\end{gather*}

\subsection{}

\subsection{}