\documentclass[10pt]{jsarticle}

%% パッケージ
\usepackage{titlesec} % sectionの見た目を編集
\usepackage{color} % 文字・背景の色を編集
\usepackage{url} % URLを出力
\usepackage{amsmath,amssymb} % 数式全般
\usepackage{mathtools}
\usepackage[dvipdfmx]{graphicx} % 図の挿入
\usepackage{tikz} % 図の描画
\usetikzlibrary{intersections,calc,arrows.meta} % TikZのライブラリを追加

%% sectionの修飾
% 四角+下線
\titleformat{\section}[block]
{}{}{0pt}
{
  \colorbox{black}{\begin{picture}(0,10)\end{picture}}
  \hspace{0pt}
  \normalfont \Large\bfseries
  \hspace{-4pt}
}
[
\begin{picture}(100,0)
  \put(3,18){\color{black}\line(1,0){300}}
\end{picture}
\\
\vspace{-30pt}
]
\renewcommand{\thesubsection}{\textbf{問題\Roman{subsection}}}
\renewcommand{\thesubsubsection}{\textbf{問\arabic{subsubsection}}}


\setcounter{tocdepth}{3}
\begin{document}

\title{東北大学工学部編入学試験過去問解答}
\author{comimome \\ \url{https://github.com/comimome/}}
\date{\today}
\maketitle

\tableofcontents%目次
\newpage

\section{はじめに}

\newpage

\section{令和5年度 数学}

\subsection{}

\subsubsection{}
ベクトル $\overrightarrow{\mathrm{AB}}$ を求め,その大きさを計算する.$\overrightarrow{\mathrm{AB}}$ は
\begin{align*}
  \overrightarrow{\mathrm{AB}} = \begin{pmatrix}4 \\ 5 \\ -2 \end{pmatrix} - \begin{pmatrix}2 \\ 3 \\ -1 \end{pmatrix} = \begin{pmatrix}2 \\ 2 \\ -1 \end{pmatrix}
\end{align*}
となる.よって
\begin{align*}
  \lvert\overrightarrow{\mathrm{AB}}\lvert = \sqrt{2^2 + 2^2 + (-1)^2} = \sqrt{10}
\end{align*}
である.


\subsubsection{}
\begin{tikzpicture}
  \coordinate (A) at (2,3,-1) node [below left] at (A) {A};
  \coordinate (B) at (4,5,-2) node [below right] at (B) {B};
  \draw [name path=AB][thick](A)--(B);
  \draw [name path=cir_A][thick](A) circle[radius=1];
  \draw [name path=cir_B][thick][dotted](B) circle[radius=3];
  \path [name intersections={of=AB and cir_A, by={C}}];
  \fill (C) node [above] {C};
  \foreach\P in {A,B,C} \fill[black](\P)circle(0.06);
\end{tikzpicture}


\subsubsection{}
\subsubsection{}

\subsection{}

\subsection{}

\end{document}